
%%%
%%% CHAPTER
%%%
\chapter{Upscaling Formulation Model}\label{Chapter:UpscalingFormulationModel}

%%% Summary of the Chapter
\begin{adjustwidth}{1cm}{1cm}
   {\it The main aim of this chapter is to describe the variety of determininstic and stochastic upscaling models and ...}
\medskip
\end{adjustwidth}


\section{Definitions}\label{Chapter:UpscalingFormulationModel:Section:Definitions}

Given an open bounded domain $\Omega\subset\mathbb{R}^{d}$ with prescribed border $\partial\Omega$. Let's also define a space of all $\left(\text{d}\times\text{d}\right)$ real symmetric matrices\footnote{A matrix $\underline{\underline{A}}$ is said to be symmetric if $\underline{\underline{A}} = \underline{\underline{A}}^{T}$.}, $M_{d}^{\mathcal{S}}\left(\mathbb{R}\right)$. A subset of this space is the space of all real symmetric positive-definite matrices, $M_{d}^{+}\left(\mathbb{R}\right)$, such that,
\begin{displaymath}
  M_{d}^{+}\left(\mathbb{R}\right) \subset M_{d}^{\mathcal{S}}\left(\mathbb{R}\right),
\end{displaymath}

\begin{shaded}
  \begin{center} {\bf A few important definitions of positive-definite matrices}\end{center}
  %
\begin{defn}{Symmetric:}\label{Def:Symmetric} A squared $d\times d$ matrix $\underline{\underline{A}}$ is said to be {\it symmetric} if $\underline{\underline{A}} = \underline{\underline{A}}^{T}$.
\end{defn}
%
\begin{defn}{Positive-Definite Matrix:}\label{Def:PositiveDefinite} A squared $d\times d$ symmetric matrix $\underline{\underline{A}}$ is {\it positive definite} if $\forall \underline{z}\in\mathbb{R}^{d}$ $\left(\text{where }\underline{z}\text{ is a non-zero column vector}\right)$,
  \begin{equation}
    \underline{z}^{T}\underline{\underline{A}}\; \underline{z} > 0
  \end{equation}
\end{defn}
\medskip

\begin{center}{\bf General Properties:}\end{center}
Suppose that the matrix $\underline{\underline{A}}$,
\begin{displaymath}
  \underline{\underline{A}} = \left[a_{ij}\right] =
\begin{pmatrix}
  a_{11}    &   a_{12}     & \cdots   & a_{1d}   \\
  a_{21}    &   a_{22}     & \cdots   & a_{2d}   \\
  \vdots   &    \vdots   & \ddots   & \vdots  \\
  a_{d1}    &   a_{d2}     & \cdots   & a_{dd}   \\
\end{pmatrix},
\end{displaymath}
is a positive-definite matrix, then
\begin{enumerate}[i)]
%
   \item\label{prop1} The $p\times p$ $\left(\text{with }1\le p\le d\right)$ submatrix $\underline{\underline{A}}_{p}$, 
\begin{displaymath}
\begin{pmatrix}
  a_{11}    &   a_{12}     & \cdots   & a_{1p}   \\
  a_{21}    &   a_{22}     & \cdots   & a_{2p}   \\
  \vdots   &    \vdots   & \ddots   & \vdots  \\
  a_{p1}    &   a_{p2}     & \cdots   & a_{pp}   \\
\end{pmatrix},
\end{displaymath}
is also {\it positive definite};
%
   \item\label{prop2} The $d$ eigenvalues of $\underline{\underline{A}}$, $\lambda_{1}, \lambda_{2}, \cdots, \lambda_{d}$ are positive;
%
   \item\label{prop3} There is a {\it unique} decomposition of $\underline{\underline{A}}$ ({\it Cholesky decomposition}),
     \begin{equation}
       \underline{\underline{A}} =  \underline{\underline{L}}\underline{\underline{L}}^{T},
     \end{equation}
     where $\underline{\underline{L}}$ is a lower triangular matrix, 
     \begin{displaymath}
       \underline{\underline{L}} = \left[l_{ij}\right] =
\begin{pmatrix}
  l_{11}    &   l_{12}     & \cdots   & l_{1d}   \\
  l_{21}    &   l_{22}     & \cdots   & l_{2d}   \\
  \vdots   &    \vdots   & \ddots   & \vdots  \\
  l_{d1}    &   l_{d2}     & \cdots   & l_{dd}   \\
\end{pmatrix}.
     \end{displaymath}
%     
     \item\label{prop4} There is also a unique decomposition,
     \begin{equation}
       \underline{\underline{A}} =  \underline{\underline{B}}\underline{\underline{B}}^{T},
     \end{equation}
     where $\underline{\underline{B}} = \underline{\underline{A}}^{1/2}$
%     
\end{enumerate}

%
\end{shaded}


Such domain can be characterised by a second-order symmetric tensor field, $\PermTensor{\mathcal{K}}$, with $\underline{x}$ being the spatial coordinates. 
and 
\begin{displaymath}
  \PermTensor{\mathcal{K}} \subset M_{d}^{+}\left(\mathbb{R}\right).
\end{displaymath}
The tensor $\PermTensor{\mathcal{K}}$ containing components
\begin{displaymath}
  \PermComp{\mathcal{K}},
\end{displaymath}
is defined on a probability space $\left(\Theta, \mathcal{F}, \mathcal{P}\right)$, where
\begin{itemize}
   \item $\Theta$: sample space which is the set of all possible outcomes;
    \item $\mathcal{F}$: set of events, where each event is a set containing zero or more outcomes;
    \item $\mathcal{P}$: function of the assignment of probabilities to the event.
\end{itemize}

\[
\begin{bmatrix}
  \mathcal{K}_{11}  & \mathcal{K}_{12}  & \mathcal{K}_{13} \\
  \mathcal{K}_{21}  & \mathcal{K}_{22}  & \mathcal{K}_{23} \\
  \mathcal{K}_{31}  & \mathcal{K}_{32}  & \mathcal{K}_{33} \\
\end{bmatrix}
=
\begin{bmatrix}
  \mathcal{K}_{11}  & \mathcal{K}_{21}  & \mathcal{K}_{31} \\
  \mathcal{K}_{12}  & \mathcal{K}_{22}  & \mathcal{K}_{32} \\
  \mathcal{K}_{13}  & \mathcal{K}_{23}  & \mathcal{K}_{33} \\
\end{bmatrix}
\]
